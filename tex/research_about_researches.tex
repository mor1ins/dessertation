
\documentclass[12pt]{article}
\usepackage{lingmacros}
\usepackage{tree-dvips}

\usepackage [T2A] {fontenc}   % Кириллица в PDF файле
\usepackage [utf8] {inputenc} % Кодировка текста: utf-8
\usepackage [russian] {babel} % Переносы, лигатуры

\begin{document}

\section*{Введение}

\section{Анализ статей}

\subsection{Linear-quadratic blind source separating structure for removing show-through in scanned documents}
Сравнение алгоритма Hosseini and Deville с собственной разработанной моделью
для удаления просвечивающейся сквозь лист информации.


\subsection{A local linear level set method for the binarization of degraded historical document images}
Представлен крутой метод для бинаризации текста (еще я там видел применение метода Sauvola) Но есть ограничения:
\begin{list}{}{}
    \item изображения хоть и содержат артефакты, но не содержат разницы освещения
    \item сам текст достаточно высокого качества
    \item их метод иногда отрабатывает хуже Sauvola
\end{list}


\subsection{Coupled snakelets for curled text-line segmentation from warped document images}
Метод сегментации изображений фон и строки в повернутых и изогнутых документах, возможно, снятых под углом.
Метод справляется хорошо, но интересно как это использовать для восстановления формы документа. + использовались
хорошо банризованные изображения.


\subsection{Clutter noise removal in binary document images}
Авторы удаляют клюттер из изображений. Результат выглядит очень и очень достойным. Будет полезным для восстановления
испорченных изображений (например, при бинаризации плохо пропечатанных документов, либо в плохом освещении).


\subsection{Efficient skew detection of printed document images based on novel combination of enhanced profiles}
Поворачивают изображения, причем бинаризованные и явно отсканированные (они ровные). Интересно, но
мало применимо в моей работе. Работа \textbf{Skew detection and correction based on an axes-parallel bounding box}
в ту же степь.



\subsection{Degraded Historical Documents Images Binarization Using a Combination of Enhanced Techniques}
Хорошо бинаризуют, но аж 3 метода используют.


\subsection{Adaptive binarization of severely degraded and non-uniformly illuminated documents}
Супер крутая статья. Начиная с того, что хорошо рассказывают про методы бинаризации, а также сравнивают их,
заканчивая тем, что представляют алгоритм, который дает еще более кайфовые результаты. Точно буду использовать
в работе. Как минимум, надо вставить описание методов. 


\subsection{Resolution enhancement of textual images via multiple coupled 
dictionaries and adaptive sparse representation selection}
Рассказывают про методы интерполяции, а также про свою разработку на основе словарей. Интересно в качестве сравнения методов.


\subsection{Skeletonization Algorithm for Binary Images}
Похоже на простую (хоть и теративную) реализацию скелетинизации. \textbf{A new ring radius transform-based thinning method 
for multi-oriented video characters} про сложную с восстановлением скелета.


\subsection{Restoring camera-captured distorted document images}
Поворот бинарного изображения с восстановлением перспективы. Однозначно будет полезно.


\subsection{An experimental comparison of min-cut/max- flow algorithms for energy minimization in vision}
Восстановление и сегментация изображения, сложно, но интересно.


\subsection{Document image binarization based on topographic analysis using a water flow model}
Результаты обработки достаточно интересны, а алгоритм не выглядит сложным. 


\subsection{Text and non-text separation in offline document images: a survey}
Объяснение кучи методов сегментации.


\subsection{Binarization of degraded document images based on contrast enhancement}
Еще один метод бинаризации. Вроде простой.


\subsection{A restoration method for distorted comics to improve comic contents identification}
Крутая обработка комиксов. Цветокоррекция, повороты и восстановления страницы. топчик.


\subsection{Bleed-through cancellation in non-rigidly misaligned recto–verso 
archival manuscripts based on local registration}
Очень похоже на паралелльную обработку изображения для удаления просвеченного текста. Надо разбираться.


\subsection{Coarse-to-fine document localization in natural scene 
image with regional attention and recursive corner refinement}
Есть рекурсивный алгоритм обнаружения границы, выглядит интересно. Но сложно.


\subsection{Plug-and-play approach to class-adapted blind image deblurring}
Убираем размытие на лицах и текста. Что-то сложное, да еще и с обучением, но результат хорош.
Надо разбираться, если такая задача станет.


\subsection{A comparison of local features for camera-based document image retrieval and spotting}
Восстановление изображения из кучи маленьких. Сравнение алгоритмов. 



\section{Погреб для нейронок}


\subsection{Text recognition in multimedia documents: a study of two neural-based OCRs using and avoiding character segmentation}
Выделяют отдельные символы нейронкой.


\subsection{Fully convolutional network with dilated convolutions for handwritten text line segmentation}
Сегментация изображения на линии.


\end{document}